\documentclass[aspectratio=169,t]{beamer}
\usepackage[utf8]{inputenc}
\usepackage[T1]{fontenc}
\usepackage[export]{adjustbox}
\usepackage{amssymb}
\usepackage{listings}
\lstloadlanguages{[x86masm]Assembler}


\title{Programmierung 1}
\date{SS 2020}
\author[PWD]{Prof. Dr.-Ing. Piotr Wojciech Dabrowski}
\titlegraphic{Bilder/logo.png}

\usepackage{HTWBeamerTemplate/beamerthemeHTW}
%\setbeameroption{show notes on second screen}

\subtitle{1: Allgemeines}
\addbibresource{sources_01.bib}
\begin{document}

\setbeamertemplate{footline}[first]

\begin{frame}[noframenumbering]
    \titlepage
    \begin{textblock}{10}(4.75,15)
        \cite{logo}
    \end{textblock}
\end{frame}

\setbeamertemplate{footline}[presentationbody] 

\begin{frame}{Vorstellung}
	\begin{itemize}
		\item Kontakt:
		\begin{itemize}
			\item Piotr.Dabrowski@htw-berlin.de
			\item Sprechstunde: Montag 10-11, WH C 614
		\end{itemize}
		\item Background
		\begin{itemize}
			\item Medizinische Biotechnologie + Informatik, dann Bioinformatik
			\item Analyse von Next Generation Sequencing-Daten
			\item Aufbau der Bioinformatik Core-Facility am Robert Koch-Institut
			\item Potentielles Wiedersehen: Gesundheitsinformatik
		\end{itemize}
	\end{itemize}
\end{frame}

\begin{frame}{Programmierung}
	Programmierung: Computern erklären, was sie tun sollen. \\Zu beachten: Computer
	\begin{itemize}
		\item sind sehr schnell
		\item nehmen alles wörtlich
		\item denken nicht mit
	\end{itemize}
	Aufgabe der Programmierer: Komplexe Problemlösungen so in kleine Schritte aufbrechen, dass ein Computer sie durchführen kann = Algorithmen schreiben.\\
	Dazu notwendig:
	\begin{itemize}
		\item Die richtige Art, zu denken
		\item Viel Übung - Können ist hier wichtiger als Wissen!
	\end{itemize}
\end{frame}

\begin{frame}{Beispiel}
	Sagen Sie mir, wie ich die Tür öffnen kann!\\Ich verstehe:
	\begin{itemize}
		\item Drehen nach <links, rechts> um <Grad>
	\end{itemize}
\end{frame}

\begin{frame}[allowframebreaks]{Quellenangaben}
    \printbibliography
\end{frame}

\include{lizenz}

\end{document}
