\documentclass[aspectratio=169,t]{beamer}
\usepackage[utf8]{inputenc}
\usepackage[T1]{fontenc}
\usepackage[export]{adjustbox}
\usepackage{amssymb}
\usepackage{listings}
\lstloadlanguages{[x86masm]Assembler}


\title{Programmierung 1}
\date{SS 2020}
\author[PWD]{Prof. Dr.-Ing. Piotr Wojciech Dabrowski}
\titlegraphic{Bilder/logo.png}

\usepackage{HTWBeamerTemplate/beamerthemeHTW}
%\setbeameroption{show notes on second screen}

\subtitle{1: Allgemeines}
\addbibresource{sources_01.bib}
\begin{document}

\setbeamertemplate{footline}[first]

\begin{frame}[noframenumbering]
    \titlepage
    \begin{textblock}{10}(4.75,15)
        \cite{logo}
    \end{textblock}
\end{frame}

\setbeamertemplate{footline}[presentationbody] 

\begin{frame}{Vorstellung}
	\begin{itemize}
		\item Kontakt:
		\begin{itemize}
			\item Piotr.Dabrowski@htw-berlin.de
			\item Sprechstunde: Montag 10-11, WH C 614
		\end{itemize}
		\item Background
		\begin{itemize}
			\item Medizinische Biotechnologie + Informatik, dann Bioinformatik
			\item Analyse von Next Generation Sequencing-Daten
			\item Aufbau der Bioinformatik Core-Facility am Robert Koch-Institut
			\item Exkurse in BMF, BMI $\rightarrow$ Projekte bei Interesse möglich
			\item Potentielles Wiedersehen: Gesundheitsinformatik
			\item Seit WS 2019/2020 an der HTW, seit SS 2020 Programmierung 1
		\end{itemize}
	\end{itemize}
\end{frame}

\begin{frame}{Programmierung}
	Programmierung: Computern erklären, was sie tun sollen. \\Zu beachten: Computer
	\begin{itemize}
		\item sind sehr schnell
		\item nehmen alles wörtlich
		\item denken nicht mit
	\end{itemize}
	Aufgabe der Programmierer: Komplexe Problemlösungen so in kleine Schritte aufbrechen, dass ein Computer sie durchführen kann = \textbf{Algorithmen} schreiben.\\
	Dazu notwendig:
	\begin{itemize}
		\item Die richtige Art, zu denken
		\item Viel Übung - Können ist hier wichtiger als Wissen! $\rightarrow$ Vorlesungsaufbau (...braucht man die Vorlesung überhaupt?)
	\end{itemize}
\end{frame}

\begin{frame}{Beispiel}
	Sagen Sie mir, wie ich die Tür öffnen kann!\\Ich verstehe:
	\begin{itemize}
		\item Drehen nach <links, rechts> um <wieviel> Grad
		\item Gehen nach <vorne, hinten> um <wieviel> Schritt
		\item Arm auf der <rechten, linken> Seite um <wieviel> Grad <heben, senken>
		\item Greifen mit der <rechten, linken> Hand
		\item Loslassen mit der <rechten, linken> Hand
	\end{itemize}
	Jeder, der den Ball hat, sagt eine Anweisung\\Freiwillige Hilfe: Algorithmus aufschreiben.
\end{frame}

\begin{frame}{Von Lösiungsidee zu Programm}
	Lästige Schreibweise im Beispiel $\rightarrow$ Konventionen, wie Anweisungen geschrieben werden = \textbf{Syntax}. Beispiel:
	\begin{itemize}
		\item ``nach'', ``um'', ``Grad'' etc. weglassen: ``Drehen nach links um 10 Grad'' $\rightarrow$ ``Drehen links 10''
		\item Zur Lesbarkeit ein paar Trennzeichen einführen: ``Drehen links 10'' $\rightarrow$ ``Drehen(links, 10)''
		\item Klar machen, wann ein Befehl beendet ist und der nächste anfängt: ``Drehen(links, 10)'' $\rightarrow$ ``Drehen(links, 10);''
	\end{itemize}
	$\rightarrow$ Übersetzung des Algorithmus in Programmcode
	\\Vokabular zum einfacheren Unterhalten: \textbf{Funktionen} und \textbf{Argumente}
\end{frame}

\begin{frame}{Für wen machen wir das?}
	Computer verstehen aber kein ``Drehen'' etc., sondern im Grunde nur:
	\begin{itemize}
		\item AusSpeicherHolen(vonAdresse, inRegister)
		\item Addieren(vonRegister, zuRegister, ergebnisRegister)
		\item Substrahieren(register, vonRegister, ergebnisRegister)
		\item InSpeicherSchreiben(nachAdresse, vonRegister)
		\item SpringeWennRegisterNull(register, sprungWeite)
	\end{itemize}
	Das reicht aber (siehe Computeraufbau)! Beispiele:
	\begin{itemize}
		\item Zwei Zahlen multiplizieren
		\item Bild auf Bildschirm anzeigen
	\end{itemize}
	...aber will man so programmieren?
\end{frame}

\begin{frame}{Zahlen multiplizieren}
\end{frame}

\begin{frame}{Kontrollstrukturen}
	Frage: Reichen Funktionen und Argumente, um jedes Problem zu lösen? Falls nein, was fehlt?
\end{frame}

\begin{frame}{Beispiel weiter}
	\begin{itemize}
		\item Funktioniert das nochmal (-> Startbedingungen oder Berechnungen)
		\item Für Berechnungen: Werte -> Variablen -> Datentypen etc.
	\end{itemize}
\end{frame}

\begin{frame}[allowframebreaks]{Quellenangaben}
    \printbibliography
\end{frame}

\include{lizenz}

\end{document}
